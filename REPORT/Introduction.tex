\chapter{Introduction}
Hawkes process is a particularly interesting class of stochastic process that
was introduced in the 1970s by A. G. Hawkes, notably to model the occurrence of seismic events. Since then it has been applied in diverse areas, from earthquake modeling to financial analysis. The process is characterised by a stochastic intensity vector, which represents the conditional probability density of the occurrence of an event in the future. It is point process whose defining characteristic is that it self-excites, meaning that each arrival increases the rate of future arrivals for some period of time.
\\
The report is organized as follows:
\\
Chapter 2 introduces the definitions of the counting process, point process, nonhomogeneous Poisson process and intensity function.
\\
Chapter 3 gives the definitions of Hawkes process and conditional intensity function. After that, presenting some algorithms by thinning and cluster. 
\\
Finally, in Chapter 4, we discuss the possible applications of Hawkes process and talk
about seismic events, insurance company surplus.