\chapter{Introduction}
Hawkes processes are a particularly interesting class of stochastic processes that
were introduced in the 1970s by A. G. Hawkes, notably to model the occurrence of seismic events. Since then they have been applied in diverse areas, from earthquake modeling to financial analysis. The processes themselves are characterized by a stochastic intensity vector, which represents the conditional probability density
of the occurrence of an event in the immediate future. They are point processes whose
defining characteristic is that they self-excite, meaning that each arrival increases the
rate of future arrivals for some period of time.
\\
The report is organized as follows:
\\
Chapter 2 introduces definitions of counting process, point process, nonhomogeneous Poisson process and intensity function.
\\
Chapter 3 gives definition of Hawkes process and conditional intensity function. We also provide proofs for some of the results. After that, presenting some algorithms by thinning and cluster. 
\\
Finally, in Chapter 4 we discuss possible applications of Hawkes processes. We talk
about crime data, insurance company surplus.