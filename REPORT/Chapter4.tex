\chapter{Applications}
This chapter will introduced mathematically some applications which are using the Hawkes process.

\section{Seismic Events}
In reality, the earthquakes regularly occur. To decrease the risk and damage, Alan Hawkes introduced a family of probability models to the prediction of the occurrence of large earthquakes (related to the assessment of seismic risk in space and time) is complicated by the presence of clusters of aftershock. The epidemic type aftershock sequence model was introduced. This model is a particular type of marked Hawkes process for modeling earthquake times and magnitudes. the epidemic type aftershock sequence model can be defined by its intensity

$$\lambda(t) = \lambda_{0} + \alpha \sum_{t_{i} < t} e^{\delta\kappa_{i}} e^{-\beta(t-t_{i})}.$$
where $\alpha, \beta, \delta > 0 $ are parameters, with an exponential distribution as its mark density  and $\kappa_{i} \in [0, \infty)$ denotes the magnitude of an earthquake occurring
at time $t_{i}.$
$$f(\kappa|t) = \gamma e^{-\gamma\kappa}.$$
The conditional intensity function including both marks and times is
$$\lambda(t,\kappa) =( \lambda_{0} + \alpha \sum_{t_{i} < t} e^{\delta\kappa_{i}} e^{-\beta(t-t_{i})})\gamma e^{-\gamma\kappa}.$$

 \begin{figure}[H]
 	\centering
 	\includegraphics[width=0.8\textwidth ]{Application_Seismic.PNG}
 	\caption{The number of shocks in periods of three months for an area of the North Atlantic.}
 	\label{Application_Seismic}
 \end{figure}

Seeing in Figure \ref{Application_Seismic} that the number of shocks in periods of three months for an area of the North
Atlantic resembles the stochastic intensity function of a Hawkes process.

\section{ Risk Process with Hawkes Process}
In insurance field, the risk estimation is important. Hence Stabile and Torrisi consider risk processes with non-stationary Hawkes claims arrivals. They introduce the following risk model for the surplus process (risk process)
$$ U(t,x) = x + ct - \displaystyle\sum_{i=1}^{N^{*}(t)}Z_{i}$$
where ${N(t), t \geq 0} $is the number of points of a non-stationary Hawkes process, in the time interval $(0, t]$, $x$,$ c > 0$ and $\{Z_{i}, i = 1, 2, . . .\}$ are the same as for the classic model.

 \begin{figure}[H]
 	\centering
 	\includegraphics[width=0.8\textwidth ]{SurplusProcess.PNG}
 	\caption{Behavior of the surplus process.}
 	\label{Surplus_Process}
 \end{figure}
 The interpretation of the risk model is the following: the
 standard claims which occur according to the immigrant-points trigger claims according
 to the branching structure. Typically, the fertility rate of an immigrant is taken to
 be monotonic decreasing, meaning that the claim number process has a self -exciting
 structure in which recent events affect the intensity of claim occurrences more than
 distant ones.
